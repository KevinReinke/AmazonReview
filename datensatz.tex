\documentclass{article}

\usepackage[ngerman]{babel}
\usepackage[utf8x]{inputenc}
\usepackage{listings}
\usepackage{url}
\usepackage{fontspec}

\newfontfamily\Consolas{Consolas}
\lstset{basicstyle=\Consolas}

\begin{document}
\section{Datensatz}
Der in dieser Arbeit verwendete Datensatz besteht aus Reviews des Onlinehändlers Amazon aus der Zeitspanne von Mai 1996 bis Juni 2014. Es wird hier ein spezieller Teildatensatz verwendet, welcher lediglich Reviews aus der Kategorie Elektronik enthält. Dieser hat mit 7,8 Millionen Reviews von etwa 4,2 Millionen Nutzern eine Größe von 4,7 Gigabyte. Er ist vollständig im JSON-Format gehalten. Ein Beispielreview ist nachfolgend dargestellt:

\begin{lstlisting}[breaklines=true]
{ 
  "reviewerID": "A2SUAM1J3GNN3B", 
  "asin": "0000013714", 
  "reviewerName": "J. McDonald", 
  "helpful": [2, 3], 
  "reviewText": "I bought this for my husband who plays the piano. He is having a wonderful time playing these old hymns. The music is at times hard to read because we think the book was published for singing from more than playing from. Great purchase though!", 
  "overall": 5.0, 
  "summary": "Heavenly Highway Hymns", 
  "unixReviewTime": 1252800000, 
  "reviewTime": "09 13, 2009" 
}
\end{lstlisting}

Die Datensätze bestehen aus der ID des Reviewers (reviewerID), der Produkt-ID (asin), dem Namen des Reviewers (reviewerName), einer Liste mit zwei numerischen Werten, die die hilfreich-Bewertungen repräsentieren (helpful), dem Inhalt der Review (reviewText), der Produktbewertung (overall), der Überschrift bzw. Zusammenfassung des Reviews (summary), dem Datum der Review (reviewTime) und dem Datum im Unix-Format (unixReviewTime). Aufgrund der gegebenen Reviewer-ID lassen sich so auch neben den Reviews Rückschlüsse über die einzelnen Reviewer ziehen.\\
Der Datensatz ist verfügbar unter \url{http://jmcauley.ucsd.edu/data/amazon/}.
\end{document}